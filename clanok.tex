% Metódy inžinierskej práce

\documentclass[10pt,twoside,slovak,a4paper]{article}

\usepackage[slovak]{babel}
%\usepackage[T1]{fontenc}
\usepackage[IL2]{fontenc} % lepšia sadzba písmena Ľ než v T1
\usepackage[utf8]{inputenc}
\usepackage{graphicx}
\usepackage{url} % príkaz \url na formátovanie URL
\usepackage{hyperref} % odkazy v texte budú aktívne (pri niektorých triedach dokumentov spôsobuje posun textu)

\usepackage{cite}
%\usepackage{times}

\pagestyle{headings}

\title{Cloud gaming\thanks{Semestrálny projekt v predmete Metódy inžinierskej práce, ak. rok 2022/23, vedenie: Ing. Zuzana Špitálová}}

\author{Milan Marcinčo\\[2pt]
	{\small Slovenská technická univerzita v Bratislave}\\
	{\small Fakulta informatiky a informačných technológií}\\
	{\small \texttt{xmarcinco@stuba.sk}}
	}

\date{\small 13. december 2022}



\begin{document}

\maketitle

\begin{abstract}
Vývoj v oblasti technológií a informatiky ako takej napreduje ohromnou rýchlosťou už dlhé roky, a to bez akýchkoľvek náznakov približovania sa k stropu. Ruka v ruke s ním rastie aj náročnosť počítačových hier a požiadavky na parametre hardvéru. Staršie generácie procesorov a grafických kariet už nedokážu poskytovať dostatočný výpočtový výkon pre nové herné tituly a vyžadujú si častú výmenu za nový, moderný kus. Jedným z možných riešení tohto problému je hranie v cloude (cloud gaming), čím sa zaoberá aj tento článok. Jedná sa o konceptovo veľmi jednoduchý, ale realizačne veľmi zložitý prístup. S aktuálne dostupnými technológiami a ich rozšírením má hranie v cloude mnoho zraniteľných miest a prekážok na ceste k dokonalosti.
\end{abstract}



\section{Úvod}

\subsection{Čo je to cloud gaming?}

Cloud gaming alebo všeobecnejšie cloud computing je veľmi atraktívne, moderné riešenie pre hráčov dnešnej doby, vo forme predplatnej služby. Jedným z hlavných faktorov pri výbere nových komponentov hernej zostavy je okrem výkonu aj cena, ktorá vďaka globálnemu nedostatku čipov v nedávnej minulosti niekoľkokrát prekonala svoje predošlé maximá. Oblasť, v ktorej služba hrania v cloude vyniká, je práve pomer ceny, výkonu a aktuálnosti hardvéru, na ktorom herné tituly bežia. Tú si na svoju zodpovednosť berie poskytovateľ služby, čo je pre koncového používateľa veľkou výhodou.



\subsection{Ako to funguje?}

Hlavná myšlienka, na ktorej celé hranie v cloude stojí, je presun výkonovo náročných výpočtových operácií z používateľovho osobného počítača do cloudu. Každému hráčovi sú v čase hrania priradené hardvérové prostriedky, vrátane procesorových jadier, pamäte RAM, jadier grafickej karty a úložného priestoru, zvyčajne typu SSD. Tieto prostriedky sú využívané na všetky výpočty potrebné k behu hry a enkódovanie výstupného obrazu. Takto zakódované video je potom po internetovej sieti v reálnom čase streamované do počítača koncového používateľa, kde je následne zobrazené. Rovnakým spôsobom je prenášaný aj vstup z klávesnice, myši a iných vstupných periférií do cloudu, kde je spracovávaný a kde ovplyvňuje priebeh hry a jej výstupný obraz.

V praxi to znamená, že počítač na strane používateľa je vyťažovaný len dekódovaním výstupného obrazu hry a spracovávaním a odosielaním vstupov, na čo je vo výsledku potrebný len zlomok výpočtového výkonu. To dovoľuje používateľovi použiť pri hraní v cloude takmer akékoľvek zariadenie, aj so staršími a pomalšími komponentami.\cite{7182690}



\section {Analýza}



\subsection{Výhody oproti konvenčným herným zostavám\cite{10.1007/978-981-10-6620-7_71}}

\begin{itemize}

\item Odpadá nutnosť vlastniť drahý, výkonný počítač a pravidelne ho modernizovať, čo značne zníži nie len počiatočné, ale celkové náklady počas niekoľkých rokov.

\item Ukladanie progresu hry umožňuje pokračovať v hraní na rôznych zariadeniach z akéhokoľvek miesta, kde je stabilné internetové pripojenie.

\item Veľkosť úložného priestoru nehrá rolu, a tak nie je limitom pre množstvo hier, z ktorých si môže používateľ vybrať. Týmto je zároveň značne zvýšená efektivita, keďže nepotrebuje každý používateľ osobitnú kópiu celej hry, ale je zdieľaná celým datacentrom.

\item O funkčnosť a správny beh hardvéru sa stará poskytovateľ. Odpadajú tak akékoľvek starosti so servisom a zárukou hardvéru pre koncového užívateľa.

\end{itemize}



\subsection{Nevýhody (slabé stránky cloudu)\cite{10.1007/978-981-10-6620-7_71}}

\begin{itemize}

\item V prípade neoptimálnej stability internetového pripojenia je zážitok z hier, najmä akčného multiplayer typu, viditeľne zníženy. Keďže sa všetky výpočty odohrávajú mimo počítača koncového užívateľa, každá akcia má nejakú odozvu, ktorá je v prípade týchto typov hier značné citeľná.

\item Stabilné internetové pripojenie je stále v mnohých častiach sveta nedostupné, alebo je veľmi nákladné. V tomto prípade hranie v cloude už nie je múdrou alternatívou, a častokrát ani možné.

\item Závislosť na prevádzkovateľovi služby je taktiež možnou nevýhodou. Akýkoľvek výpadok alebo útok na poskytovateľa znamená pre koncového používateľa neschopnosť službu využívať.

\end{itemize}



\subsection{Dostupné platformy a ich porovnanie}

GeForce Now, Google Stadia, Xbox Cloud Gaming, PlayStation Now, Amazon Luna



\subsubsection{Kvalita streamovania}

\subsubsection{Knižnica dostupných hier}

\subsubsection{Dostupnosť v krajinách}

\subsubsection{Unikátne vlastnosti jednotlivých platforiem}

\subsection{Výzvy a prekážky}



\section{Zhodnotenie}

\subsection{Cieľová skupina používateľov}

\subsection{Cloud gaming ako budúcnosť herného priemyslu?}



\section{Záver}



%\acknowledgement{Ak niekomu chcete poďakovať\ldots}


% týmto sa generuje zoznam literatúry z obsahu súboru literatura.bib podľa toho, na čo sa v článku odkazujete
\bibliography{literatura}
\bibliographystyle{plain} % prípadne alpha, abbrv alebo hociktorý iný
\end{document}
